\begin{figure} \centering
\begin{subfigure}[m]{0.42\textwidth} \centering
     \tikz[scale=0.66]{
	\defrect{(0,0)}{(5.5,2)}{$a$}{$b$}
	\defrect{(0,0)}{(1.8,3.2)}{ }{ }
	\defrect{(2.7,-1)}{(3.55,2.6)}{ }{ }
	\defrect{(4.15,-1.3)}{(5.5,2)}{ }{ }
     }
\caption{Прямоугольники, независимые \\ с $a \ldots b$}
\label{fig:rectCases}
\end{subfigure}\ \ \ 
\begin{subfigure}[m]{0.5\textwidth} \centering
     \tikz[scale=0.75]{
	\defrect{(0,-1.6)}{(6.5,0)}{$a$}{$b$};
	\draw[decoration=snake,decorate] (3.5,1.8) -- (3.5,-3.5);
	\draw (3.5,-3.2) node[right]{\ $\ell$};
	\filldraw (2.5,-0.8) circle[radius=0.8mm] node[left]{$P$}
		(4.2,-0.35) circle[radius=0.8mm] node[right]{$Q$};
     }
\caption{Вертикальная линия, не пересекающая ни \\
	один из прямоугольников набора. Две точки, \\
	соответствующие прямоугольнику $a \ldots b$}
\label{fig:rectPQ}
\end{subfigure}
     \caption{Доказательство Леммы~\ref{lm:optPlusBound}}
\end{figure}