\subsection{Более практичная оценка снизу}

Рассмотрим пару $(s_i,i)$ из набора поисковых запросов. Упорядочим все остальные точки $(s_j,j)$, $j < i$ по второй координате и соединим их $y$-монотонной ломаной сверху вниз, смотреть Рисунок~\ref{fig:vertInt}.

\begin{figure}[h] \centering
\definecolor{specblue}{RGB}{100,15,190}
\newcommand{\dts}{\draw[very thick,->,specblue]}
\tikz[scale=1.12]{
	\draw[->,thick] (-1.5,0) -- (5,0);
	\draw[->,thick] (0,-1.5) -- (0,4.5);
	\coordinate (s) at (2.5,3.25);
	\defrect{(3.25,0.6)}{(s)}{ }{ }
	\defrect{(2,1)}{(s)}{ }{ }
	\defrect{(1.6,1.75)}{(s)}{ }{ }
	\defrect{(3.75,2)}{(s)}{ }{ }
	\defrect{(0.75,2.5)}{(s)}{ }{ }
	\draw (s) -- (2.5,-1.25);
	\dts (s) -- (0.75,2.5); \dts (0.75,2.5) -- (3.75,2);
	\dts (3.75,2) -- (1.6,1.75); \dts (1.6,1.75) -- (2,1);
	\dts (2,1) -- (3.25,0.6);
	\draw (s) node[above]{$s_i \in P$};
}
\caption{Подсчёт числа пересечений с вертикальной прямой}
\label{fig:vertInt}
\end{figure}

Обозначим через $J(s_i)$ количество пересечений этой ломаной с вертикальным лучом, идущим из $s_i$ вниз. Понятно, что такое число можно посчитать для любого элемента последовательности запросов.

\begin{theorem} \label{thm:optIndBound}
\begin{equation} \label{eq:optIndBound}
	\opt(P)\ \ge\ |P| + \sum\limits_{s_i} \frac{J(s_i)}{2}
\end{equation}
\end{theorem}

\begin{proof}
На каждом ребре ломаной, пересекающем вертикальный луч, построим как на диагонали прямоугольник, стороны которого параллельны осям координат. Так у каждого пересечения появится свой прямоугольник. Объединим получившиеся наборы прямоугольников, смотреть Рисунок~\ref{fig:manyInd}.

\newcommand{\rectStack}[2]{
\begin{scope}[xshift=#1 cm]
	\filldraw[draw=black,fill=black] (1.25,1.15)
		circle[radius=0.8mm] node[right]{#2};
	\defrect{(0,0)}{(1.75,0.75)}{ }{ }
	\defrect{(0.25,0)}{(1.5,-0.75)}{ }{ }
	\defrect{(1.25,-0.75)}{(0.5,-1.25)}{ }{ }
	\defrect{(0.75,-1.25)}{(1,-1.75)}{ }{ }
\end{scope}}

\begin{figure}[h] \centering
\tikz[scale=1.19]{
	\rectStack{0}{$s_1$} \rectStack{3.3}{$s_2$} \rectStack{6.6}{$s_3$}
}
\caption{Набор попарно независимых прямоугольников}
\label{fig:manyInd}
\end{figure}

Все прямоугольники в объединении, легко видеть, будут попарно независимы. Осталось лишь применить теорему~\ref{thm:optFirstBound}.
\end{proof}

\subsection{Оценка снизу через число перебежек}

Рассмотрим вершину $q$ бинарного дерева поиска $T$. Обозначим через $R(q)$ количество чередований между спусками в левое поддерево $q$ и правое поддерево $q$. Спуски в сам узел $q$ и всё, что происходит вне поддерева $q$, при этом игнорируется.

\begin{theorem} \label{thm:optPereBound}
\begin{equation} \label{eq:optPereBound}
	\opt(P)\ \ge\ \sum\limits_{q \in T} R(q).
\end{equation}
\end{theorem}

\begin{proof}
Следует из Теоремы~\ref{thm:optIndBound}.
\end{proof}

\begin{figure}[h] \centering
\begin{tabular}{|c|c|c|c|} \hline
0 & 0\,0\,0 & 0\,0\,0 & 0 \\
1 & 0\,0\,1 & 1\,0\,0 & 4 \\
2 & 0\,1\,0 & 0\,1\,0 & 2 \\
3 & 0\,1\,1 & 1\,1\,0 & 6 \\
4 & 1\,0\,0 & 0\,0\,1 & 1 \\
5 & 1\,0\,1 & 1\,0\,1 & 5 \\
6 & 1\,1\,0 & 0\,1\,1 & 3 \\
7 & 1\,1\,1 & 1\,1\,1 & 7 \\ \hline
\end{tabular}
	\caption{Bit-reversal sequence делает нижнюю оценку бессмысленно большой}
	\label{fig:brs}
\end{figure}

\section{Tango деревья}

Дерево, где у каждой вершины есть «любимый потомок» — тот, в которого происходил спуск при предыдущем запросе. Отметим у каждой вершины её любимого потомка — дерево окажется представленным виде объединения путей, смотреть Рисунок~\ref{fig:tangoTree}.

\begin{figure}[h] \centering
     \newcommand{\dcn}[1]{\filldraw[fill=black,draw=black] #1 circle[radius=0.8mm];}
     \newcommand{\dar}{\draw[very thick,->,specred]}
     \definecolor{specred}{RGB}{225,20,30}
     \tikz[scale=0.57]{
	\dar (4.5,6.5) -- (6,4.5); \dar (6,4.5) -- (5.75,2.5); \dar (5.75,2.5) -- (5,0);
	\dar (7.75,2.5) -- (8.75,0); \dar (1.25,2.5) -- (0.25,0);
	\dar (3,4.5) -- (3.25,2.5); \dar (3.25,2.5) -- (2.75,0);
%%
	\foreach \x in {0.25,1.5,2.75,4,5,6.25,7.5,8.75} {\dcn{(\x cm, 0)}}
	\foreach \x in {1.25,3.25,5.75,7.75} {\dcn{(\x cm, 2.5)}}
	\foreach \x in {3,6} {\dcn{(\x cm, 4.5)}} \dcn{(4.5,6.5)}
%%
	\draw[thick] (4.5,6.5) -- (3,4.5) -- (1.25,2.5) -- (1.5,0)
	     (6,4.5) -- (7.75,2.5) -- (7.5,0)
	     (5.75,2.5) -- (6.25,0) (3.25,2.5) -- (4,0);
     }
\caption{Tango-дерево представлено в виде объединения путей}
\label{fig:tangoTree}
\end{figure}

Каждому такому пути сопоставим дерево поиска (чтобы за $\log \log$ отправляться в нужное место пути). При смене любимого потомка у вершины нам придётся перестраивать такие деревья. Это мы умеем.