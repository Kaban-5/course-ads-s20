\section{Динамизация структур данных}

Пусть у нас есть задача поиска:

\begin{equation*}
	Q \colon X \times 2^D \to A,
\end{equation*}
где $A$~--- ответы, $D$~--- объекты, $X$~--- запросы.

\subsection{}


\begin{definition}
	Говорим, что задача поиска является decomposable, если функция $Q$ обладает следующим свойством:

	\begin{equation*}
		Q(x, D \cup D^\prime) = Q(x, D) \blacklozenge Q(x, D^\prime),
	\end{equation*}
	где $\blacklozenge$ означает, операцию, которая быстро считается.
\end{definition}


\begin{exmpl}
	Пусть мы хотим в каком-то множестве точек узнавать ближайшего соседа от точки запроса. Ясно, что мы можем узнать ближайшего соседа в множестве $D$, ближайшего соседа в множестве $D^\prime$ и взять из них того, что ближе.
\end{exmpl}

Предположим, что у нас есть такая функция $Q$. Давайте сформулируем задачу:

Итак, пусть существует структура данных, которая умеет только хранить и выдавать ответ на запрос, со следующими свойствами:

\begin{itemize}
	\item В ней $n$ объектов;
	\item Она занимает $S(n)$ памяти;
	\item Её можно построить за $P(n)$;
	\item Она отвечает на вопрос за $Q(n)$.
\end{itemize}


\begin{task}
	Мы хотим построить структуру данных, которая обладает следующими свойствами:

	\begin{itemize}
		\item Она занимает $S^\prime(n) = O(S(n))$ памяти;

		\item Она строится за $P^\prime(n) = O(P(n))$;

		\item Она отвечает на вопрос за $Q^\prime(n) = O(\log n\cdot Q(n))$;

		\item Вставка занимает $I^\prime(n) = O\left(\log n\frac{P(n)}{n}\right)$.


	\end{itemize}

	В данной задаче имеется ввиду амортизированное время для запроса и вставки.
\end{task}

Итак, давайте строить. Разбиваем наши элементы на на логарифмическое количество уровней. Теперь у нас имеется $\log n $ уровней, которые называются $L_0,\ldots,L_l$:
\begin{itemize}
	\item $L_0$: $\varnothing$ или структура с $1$ элементом

	\item $L_1$: $\varnothing$ или структура с $2$ элементами

	\item ...

	\item $L_i$: $\varnothing$ или структура с $2^{i-1}$ элементами

	\item ...

	\item $L_l$: $\varnothing$ или структура с $2^{l-1}$ элементами
\end{itemize}

Запрос делаем следующим образом:


\begin{tabular}{|p{4cm}|}
	\hline
	Query(x):

	$\quad$$a \coloneqq E$ (ответ на $\varnothing$)

		$\quad$for $i = 0$ to $l$

	$\quad$$\quad$if $L_i \ne \varnothing$

	$\quad$$\quad$$\quad$$a \coloneqq a\blacklozenge Q(x,L_i)$

	$\quad$return $a$ \\
	\hline
\end{tabular}


Ясно, что на запрос мы потратим времени
\begin{equation*}
	\sum_{i = 0}^{l-1}Q(2^i) < l\cdot Q(n) = O(\log n)Q(n).
\end{equation*}


\begin{remark}
	Если $Q(n)$~--- большое, то есть $Q(n) > n^\varepsilon$ при каком-то $\varepsilon > 0$, то время на запросе~--- это $O(Q(n))$.
\end{remark}

Вставку делаем следующим образом:


\begin{tabular}{|p{5cm}|}
	\hline
	Insert(x):

	$\quad$Find $\min k: L_k = \varnothing$

	$\quad$build $L_k \coloneqq \{x\}\cup \bigcup_{i < k} L_i$

	$\quad$for $i = 0$ to $k-1$

	$\quad$$\quad$destroy $L_i$ \\

	\hline
\end{tabular}

Ясно, что build происходит за $P\left(2^k\right)$.


Мы хотим, чтоб цена за одну вставку была

\begin{equation*}
	I^\prime(n) = O(\log n )\frac{P(n)}{n}.
\end{equation*}

\begin{equation*}
	I^\prime(n) = O(\log n )\frac{P(n)}{n} = \sum_{i = 0}^{\log n}\frac{P(2^i)}{2^i}.
\end{equation*}

Ясно, что за $n$ вставок у нас получится

\begin{equation*}
	\sum_{i = 0}^{\log n}P\left(2^i\right)\frac{n}{2^i} = n \sum_{i = 0}^{\log n}\frac{P(2^i)}{2^i} = O(\log n)P(n).
\end{equation*}
Что и требовалось.


\begin{remark}
	Если $P(n) > n^{1 + \varepsilon}$, где $\varepsilon > 0$, то $I^\prime(n) = O\left(\frac{P(n)}{n}\right)$.
\end{remark}


\begin{task}
	Мы хотим получить тот же результат, что и в Задаче 1, только время для запроса и вставку теперь не амортизированное, а в худшем случае.
\end{task}
Теперь на каждом уровне у нас будут старые структуры и новые. То есть, на уровне $i$ будут находиться структуры $O_i^1,O_i^2,O_i^3,N_i$, снова в каждой из них $\varnothing$ или $2^i$ объектов. Также ещё мы хотим, чтоб выполнялось:

\begin{itemize}
	\item если $O_i^1 = \varnothing$, то $O_i^2 = \varnothing$
	\item если $O_i^2 = \varnothing$, то $O_i^3 = \varnothing$
\end{itemize}

От $N_i$ хотим, чтоб $N_i = \varnothing$ или $N_i$ было частичной (то есть той, которая находится в процессе построения) структура для $2^i$ объектов. Также нам надо, чтоб каждый объект был ровно в одной структуре $O$ и, может быть, в структуре $N$.

Запрос делаем следующим образом:


\begin{tabular}{|p{5cm}|}
	\hline
	Query(x):

	$\quad$$a_i = E$

	$\quad$for $i = 0$ to $l$

	$\quad$$\quad$if $Q^1_i \ne \varnothing$

	$\quad$$\quad$$\quad$$a_i = a \blacklozenge Query(x,Q^1_i)$

	$\quad$$\quad$if $Q^2_i \ne \varnothing$

	$\quad$$\quad$$\quad$$a_i = a \blacklozenge Query(x,Q^2_i)$

	$\quad$$\quad$if $Q^3_i \ne \varnothing$

	$\quad$$\quad$$\quad$$a_i = a \blacklozenge Query(x,Q^3_i)$

	$\quad$return $a$ \\

	\hline
\end{tabular}

Очевидно, что время запроса у нас такое же.

Вставку делаем следующим образом:


\begin{tabular}{|p{9cm}|}
	\hline
	Insert(x):

	$\quad$for $i = l \ldots 1$

	$\quad$$\quad$if $O^1_{i-1} \ne \varnothing$ and $O^2_{i-1} \ne \varnothing$

	$\quad$$\quad$$\quad$do $\frac{P(2^i)}{2^i}$ шагов построения $N_i( \coloneqq O^1_{i-1}\cup O_{i-1}^2)$

	$\quad$$\quad$$\quad$if $N_i$ is complete

			$\quad$$\quad$$\quad$$\quad$destroy $O_{i-1}^1,O_{i-1}^2$

	$\quad$$\quad$$\quad$$\quad$$O^1_{i-1} \coloneqq O_{i-1}^3$

	$\quad$$\quad$$\quad$$\quad$destroy $O_{i-1}^3$


	$\quad$$\quad$$\quad$$\quad$Update(i)

		$\quad$$\quad$$\quad$$N_0 \coloneqq x$

	$\quad$$\quad$$\quad$$Update(0)$ \\

	\hline
\end{tabular}

Что такое Update(i)?


\begin{tabular}{|p{3cm}|}
	\hline
	Update(i):

	$\quad$if $O^1_i = \varnothing$

	$\quad$$\quad$$O^1_i \coloneqq N_i$

	$\quad$else if $O_i^2 = \varnothing$

	$\quad$$\quad$$O_i^2 \coloneqq N_i$

	$\quad$else $O_i^3 \coloneqq N_i$

	$\quad$destroy $N_i$ \\

	\hline
\end{tabular}


Все ошки не могут быть заняты, это доказывается по индукции, но это было оставлено в качестве упражнения.

Идея в том, что мы не делаем построение сразу, а делаем столько его шагов, сколько можем себе позволить. А значит, время вставки даже в худшем случае будет $O\left(\frac{P(n)}{n}\log n\right)$.

\subsection{}

\begin{definition}
	Говорим, что задача поиска является invertible, если функция $Q$ обладает следующим свойством:
	Если $D = D_1\cup D_2$, то

	\begin{equation*}
		Q(x,D_1) = Q(x,D) - \blacklozenge Q(x,D_2),
	\end{equation*}
	где $ - \blacklozenge$ быстро считается.
\end{definition}

Предположим, что у нас есть такая функция $Q$. Давайте сформулируем задачу:


\begin{task}
	Пусть у нас есть структура данных $M$, которая умеет вставлять и ещё другая структура данных $G$, в которую мы будем вставлять элемент, который хотим удалить из $M$. Хотим, чтоб амортизированное время удаления из $M$ равнялось $O\left(P(n)\frac{\log n}{n}\right)$.
\end{task}


\begin{tabular}{|p{5cm}|}
	\hline
	Query(x):

	$\quad$Return $Q(x,M) - \blacklozenge Q(x,G)$ \\
	\hline
\end{tabular}


Подвох заключается в том, что у нас может быть много удалённых элементов, поэтому весь наш анализ будет от количества элементов, которые когда-либо вставлялись. А мы хотим не этого.

Мы будем ждать момента, когда $|G| > \frac{1}{2}|M|$, и в это время перестраивать структуру полностью. Идея в том, что между двумя такими моментами пройдёт минимум $\frac{n}{2}$ удалений, а значит, когда мы считаем амортизированную стоимость удаления, мы можем заключить, что цена каждого удаления~--- это
$O\left(\frac{P(n)}{n}\right) + O\left(P(n)\frac{log n}{n}\right)$ (второе слагаемое~--- это вставки в структуру $G$).


\begin{tabular}{|p{5cm}|}
	\hline

	When $|G| > \frac{1}{2}|M|$

	$\quad$Build $M \coloneqq M\setminus G$

	$\quad$$\quad$$G \coloneqq \varnothing$ \\

	\hline
\end{tabular}

Также есть проблема, что глобальные перестроения могут помешать локальным, то есть для работы вставок мы тоже перестраиваем систему, и надо, чтоб нам хватило ресурсов для вставки, несмотря на то, что мы потратили что-то на удаление. Для борьбы с этим достаточно просто амортизированную стоимость удаления умножить на достаточно большую константу.


\begin{task}
	Мы хотим получить тот же результат, что и в задаче 3, только время теперь не амортизированное, а в худшем случае.
\end{task}

Для решения Задачи 4 мы поддерживаем три структуры: $M,I,G$:


\begin{tabular}{|p{8cm}|}
	\hline
	Query(x):

	$\quad$Return $Q(x) = Q(x,M)\blacklozenge Q(x,I) - \blacklozenge Q(x,G)$ \\
	\hline
\end{tabular}


В какой-то момент снова перестраиваем структуру, а именно в случае, если $|G| > \frac{1}{2}(|M| + |I|)$. Чтоб удовлетворять запросам, придётся заморозить наши структуры. Поэтому также поддерживаем ещё три структуры: $M^\prime,I^\prime,G^\prime$.

Во время работы по перестроению структуры, мы будем временно объекты, которые мы хотим удалить, вставлять в $G^\prime$, которые хотим вставить, вставляем в $I^\prime$. $M^\prime$~--- это, собственно, структура, которую мы строим. Пока мы строим:


\begin{tabular}{|p{11cm}|}
	\hline
	Query(x):

	$\quad$Return $Q(x) = Q(x,M)\blacklozenge Q(x,I) \blacklozenge Q(x,I^\prime) - \blacklozenge Q(x,G) - \blacklozenge Q(x,G^\prime)$ \\
	\hline
\end{tabular}


Когда мы делаем каждое удаления, мы будем делать $c\frac{P(n)}{n}$ (для какой-то константы $c$) шагов построения структуры $M^\prime$.


\begin{tabular}{|p{4cm}|}
	\hline
	Через $\frac{n}{c}$ шагов

	$M \coloneqq M^\prime,I \coloneqq I^\prime,G \coloneqq G^\prime$ \\
	\hline
\end{tabular}


Ясно, что время в худшем случае будет каким надо, а именно,

\begin{equation*}
	O\left(P(n)\frac{\log n}{n} \right).
\end{equation*}

\subsection{}

Наши запросы не всегда бывают invertible. Далее будет идея того, что нужно делать с теми запросами, которые не invertible.

Нам всё равно потребуется какое-нибудь условие, а именно, weak deletion in time $D(n)$. Это какая-то операция, которая даст структуре данных понять, что этого элемента не должно в ней быть, и при этом не увеличит время на запрос.

Делаем то же самое для вставок~--- поддерживаем уровни.

Когда нам нужно удалить $x$: сделаем weak deletion $x$ в каждом уровне, содержащем $x$ (+ чтоб найти эти уровни, нам нужен какой-то словарик, который по элементу называет уровни, где он содержится).

Делаем те же глобальные перестройки, то есть когда не удаленных объектов $ > \frac{1}{2}$ удалённых~--- делаем global rebuild.

Из прошлого рассуждения знаем, что

\begin{itemize}
	\item амортизированное время вставки~--- это $O\left(P(n)\frac{\log n}{n} + D(n)\right)$;
	\item амортизированное время удаления~--- это $O\left(\frac{P(n)}{n}\right)$.
\end{itemize}

Это также можно несколькими структурами сделать в худшем случае. А именно, структурами структурами: $M,S,M^\prime,S^\prime,u$. $M$~--- главная, $S$ её дублирует. Когда у нас удалений становится больше, чем половина тех элементов, которые лежат в $M$, снова начинаем глобальное перестроение. Для этого мы заморозим структуру $S$, для новых запросов заведём $ u$~--- очередь запросов, которую будем применять к $M^\prime$. $M^\prime$~--- это рабочая копия $M$ на время перестроения. Когда мы закончили, у нас $M^\prime$ будет на правах $M$ и $S^\prime$ на правах $S$. После этого надо сделать все обновления в очереди $u$. Нужно просто подобрать константы, сколько шагов построения $S^\prime$ и $M^\prime$ мы будем выполнять каждый раз, когда делаем удаление.
