\documentclass[a4paper,11pt]{article}
\usepackage{ads}

\begin{document}

\section{Об оффлайн деревьях поиска: нижняя граница времени работы, геометрическое представление}

\subsection{Основные определения и предваряющие результаты}

Пусть дано бинарное дерево поиска с $n$ ключами. Мы знаем последовательность запросов, которые зададим этому дереву: $P = \set{s_1,s_2,\ldots,s_m}$. В поисках ключей~$s_i$ мы будем бегать по дереву туда-сюда и в процессе спуска/подъёма пройдём через некоторые вершины, которые нам не нужны.

\newdefn{$E(P)$ — множество всех вершин, которые мы посетим в процессе поиска вершин с ключами из $P$. $E = P \cup X$, $X$ — множество «лишних» вершин.}

\newdefn{$\opt$ — минимальный размер $E(P)$ (обозначение множества $P$ будем опускать, и так по контексту ясно).}

\newdefn{Конечное множество $G \subset \br^2$ называется {\it\arbs,} если
     \begin{align*}
&	\forall\,a,b \in G\quad x(a)=x(b)\text{,\ \ либо\ \ }
		y(a)=y(b)\text{,\ \ либо} \\
&	\exists\text{ точка $c$ внутри прямоугольника, определённого $a$, $b$} \\
&	\text{(внутри или на границе).}
     \end{align*}}

\begin{theorem}[Доказана ранее]
	Рассмотрим последовательность запросов
	\begin{equation*}
		\set{(s_1,1),(s_2,2),\ldots,(s_m,m)} \subset \bz^2.
	\end{equation*}

	Надмножество этой последовательности может представлять из себя последовательность узлов, которые были посещены при поиске $s_1, \ldots, s_m$, в том и только том случае, если оно \arbs.
\end{theorem}

Далее мы будем рассматривать изображение последовательности запросов на плоскости, соответственно под множеством $P$ будем понимать $\set{(s_1,1),(s_2,2),\ldots,(s_m,m)}$, аналогично вторую координату приделаем к ключам вершин из множества $E$.

\newdefn{
	Пусть дано множество $P$ и его надмножество $E$. Два прямоугольника,
	определённых каждый двумя вершинами множества $P$,
	будем называть {\it независимыми,} если
\begin{enumerate}
	\item они оба не \arbs, то есть им не принадлежит ни одна точка из $E$,
	\item ни одна из вершин одного из этих прямоугольников не лежит во внутренности другого.
\end{enumerate}}

\begin{figure} \centering
\begin{subfigure}[b]{0.44\textwidth}
\centering
     \tikz[scale=0.82]{
	\defrect{(-0.5,0)}{(3,1)}{ }{ }
	\defrect{(0.5,-1)}{(1.75,2)}{ }{ }
     }
\caption{Эти прямоугольники независимы}
\end{subfigure}\hspace{0.9cm}
\begin{subfigure}[b]{0.44\textwidth}
\centering
     \tikz[scale=0.82]{
	\defrect{(0,0)}{(3.2,1.15)}{ }{ }
	\defrect{(0,0)}{(1.75,2)}{ }{ }
     }
\caption{Эти прямоугольники независимы}
\end{subfigure}\bigskip\\
\begin{subfigure}[b]{0.6\textwidth}
\centering
     \tikz[scale=0.82]{
	\defrect{(0,0.45)}{(1.5,1.85)}{ }{ }
	\defrect{(0.8,-0.7)}{(2.5,1.05)}{ }{ }
     }
\caption{Эти прямоугольники {\bf не} независимы}
\label{fig:rectNotInd}
\end{subfigure}
\caption{Примеры прямоугольников, независимых и не очень}
\label{fig:rectIndependent}
\end{figure}

\subsection{Какая-то нижняя граница}

\end{document}